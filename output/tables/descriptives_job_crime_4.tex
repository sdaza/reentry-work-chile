% latex table generated in R 3.6.0 by xtable 1.8-4 package
% Sun May  3 07:30:40 2020
\begin{table}[htp]
\footnotesize
\setlength{\tabcolsep}{10pt}
\renewcommand{\arraystretch}{1.3}
\begin{threeparttable}
\centering
\caption{Socio-demographic characteristics of women inmates \newline by four crime-employment clusters (N =207)} 
\label{tab:descriptives_job_crime_4}
\begin{tabular}{lcccc}
  \hline
Variable & Unemployed & Offenders & Self-employed & Employed \\ 
  \hline
Age* & 36.37 & 31.41 & 43.07 & 37.76 \\ 
  High school & 0.30 & 0.11 & 0.36 & 0.50 \\ 
  Number of children* & 2.42 & 2.31 & 3.10 & 2.50 \\ 
  Worked before prison & 0.42 & 0.20 & 0.74 & 0.76 \\ 
  Number of previous sentences* & 2.89 & 8.98 & 2.20 & 1.13 \\ 
  Sentence length in months* & 2.56 & 0.81 & 2.24 & 3.45 \\ 
  Dependence / abuse of drugs & 0.40 & 0.69 & 0.12 & 0.26 \\ 
  Mental health problems* & 0.07 & 0.14 & -0.02 & -0.16 \\ 
  Searched for jobs follow-up & 0.60 & 0.24 & 0.40 & 0.71 \\ 
  Prison during follow-up & 0.27 & 0.54 & 0.02 & 0.08 \\ 
   \hline
\end{tabular}
\begin{tablenotes}
\scriptsize
\item All the values are proportions except for * that are averages.
\end{tablenotes}
\end{threeparttable}
\end{table}
