% latex table generated in R 3.6.0 by xtable 1.8-4 package
% Thu Jan  9 12:50:59 2020
\begin{table}[htp]
\footnotesize
\setlength{\tabcolsep}{10pt}
\renewcommand{\arraystretch}{1.3}
\begin{threeparttable}
\centering
\caption{Descriptive statistics by employment-crime clusters \newline based on solution in Figure \ref{fig:sequences_job_crime_4} (N = 207)} 
\label{tab:descriptives_job_crime_4}
\begin{tabular}{lcccc}
  \hline
Variable & Unemployed & Offenders & Self-employed & Employed \\ 
  \hline
Age* & 35.99 & 31.44 & 43.11 & 38.46 \\ 
  High school & 0.23 & 0.18 & 0.36 & 0.54 \\ 
  Number of children* & 2.46 & 2.33 & 3.00 & 2.49 \\ 
  Worked before prison & 0.43 & 0.18 & 0.75 & 0.78 \\ 
  Number of previous sentences* & 2.62 & 9.07 & 2.09 & 1.11 \\ 
  Sentence length in months* & 2.64 & 0.68 & 2.36 & 3.55 \\ 
  Dependence / abuse of drugs & 0.41 & 0.70 & 0.11 & 0.22 \\ 
  Mental health problems* & 0.12 & 0.14 & -0.09 & -0.17 \\ 
  Searched for jobs follow-up & 0.61 & 0.23 & 0.41 & 0.73 \\ 
  Prison during follow-up & 0.25 & 0.51 & 0.02 & 0.05 \\ 
   \hline
\end{tabular}
\begin{tablenotes}
\scriptsize
\item All the values are proportions except for * that are averages.
\end{tablenotes}
\end{threeparttable}
\end{table}
