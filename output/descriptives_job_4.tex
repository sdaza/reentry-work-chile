% latex table generated in R 3.6.0 by xtable 1.8-4 package
% Thu Jan  9 12:50:59 2020
\begin{table}[htp]
\footnotesize
\setlength{\tabcolsep}{10pt}
\renewcommand{\arraystretch}{1.3}
\begin{threeparttable}
\centering
\caption{Descriptive statistics by employment clusters \newline based on solution in Figure \ref{fig:sequences_job_4} (N = 207)} 
\label{tab:descriptives_job_4}
\begin{tabular}{lcccc}
  \hline
Variable & Unemployed & Self-employed & Under-the-table & Legitimate employed \\ 
  \hline
Age* & 34.05 & 41.90 & 40.64 & 34.12 \\ 
  High school & 0.20 & 0.35 & 0.41 & 0.69 \\ 
  Number of children* & 2.39 & 3.02 & 2.64 & 2.06 \\ 
  Worked before prison & 0.31 & 0.75 & 0.68 & 0.81 \\ 
  Number of previous sentences* & 5.17 & 3.12 & 2.09 & 1.06 \\ 
  Sentence length in months* & 1.81 & 2.18 & 3.15 & 3.79 \\ 
  Dependence / abuse of drugs & 0.53 & 0.18 & 0.27 & 0.19 \\ 
  Mental health problems* & 0.14 & -0.09 & -0.04 & -0.29 \\ 
  Searched for jobs follow-up & 0.42 & 0.43 & 0.64 & 0.88 \\ 
  Prison during follow-up & 0.36 & 0.06 & 0.05 & 0.12 \\ 
   \hline
\end{tabular}
\begin{tablenotes}
\scriptsize
\item All the values are proportions except for * that are averages.
\end{tablenotes}
\end{threeparttable}
\end{table}
